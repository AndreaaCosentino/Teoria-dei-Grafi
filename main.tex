\documentclass[12pt]{report}
\usepackage{mioStile}

\pgfplotsset{compat = 1.18}

\title{Teoria dei grafi}
\author{Andrea Cosentino}
\sectionfont{\fontsize{12}{15}\selectfont}
\date{\today}
\begin{document}

\maketitle

\tableofcontents
\setlength{\columnsep}{0.8cm}
\setlength{\columnseprule}{0.2pt}
\twocolumn
\chapter{Lezione I}
\noindent
Un grafo $G = (V,E)$ è una struttura algebrica dove $V$ è l'insieme finito di vertici e $E$ è l'insieme finito di archi. Inoltre, vale che $E = [V]^2$. Dato un insieme $S$ e un qualunque intero $k \in \{2,\dots,|S|\}$, diciamo che $[S]^k$ è la collezione di tutti i sottoinsieme di $S$ formati da $k$ elementi. Per esempio, dato l'insieme $S = \{1,2,3\}$ l'insieme $[S]^2$ contiene $\{\{1,2\}, \{1,3\},\{2,3\}\}$. 

\begin{exmp}
    Un esempio di grafo $G = (V,E)$ è $V = \{1,2,3\}$ e $E = \{\{1,2\},\{1,3\}\}$.
    
\vspace{10px}
\begin{center}
\begin{tikzpicture}{}
    \node[] (A) {$1$};
    \node[below left= 2cm and 1cm of A] (B){$2$};
    \node[below right= 2cm and 1cm of A] (C){$3$};

    \draw[] (A) -- (B);
    \draw[] (A) -- (C);
\end{tikzpicture}
\end{center}
\end{exmp}

\noindent
Notare che ci concentriamo su grafi con archi non orientati.

La nomenclatura che utilizzeremo per indicare dei vertici generici è $i,j,u,v$, mentre per indicare degli archi generici è $(i,j)$. Dire $(i,j)$ implicherebbe un ordine, per evitare di scrivere $\{i,j\}$ useremo $(i,j)$ senza implicare che l'arco sia orientato.

Il numero di nodi del grafo è detto \textbf{ordine}, e corrisponde a $|V|$. Un grafo di ordine $0$ è detto grafo \textbf{vuoto}, mentre un grafo di ordine $\leq 1$ è detto grafo \textbf{banale}. Esistono solamente due grafi di ordine $2$:

\disegna{ 
    \node[]  (A1) at (-3.5,0) {$A)$};
    \node[nodo] (A) at (-3,0){};
    \node[nodo] (B) at (-1,0){};

    \node[]  (A2) at (-3.5,-2) {$B)$};
    \node[nodo] (C) at (-3,-2){};
    \node[nodo] (D) at (-1,-2){};
    \draw[] (C) -- (D);
}

\noindent
Dato un arco $e = (i,j) \in E$ diciamo che $i,j$ sono vertici incidenti all'arco $e$. Due vertici $i,j$ con $i \neq j$ tali che $(i,j) \in E$ sono detti vertici \textbf{adiacenti} in $G(V,E)$.  Se $E \equiv [V]^2$ diciamo che il grafo è \textbf{completo} oppure che è una \textbf{clique} (o cricca in italiano). Un grafo completo su $n$ vertici è chiamato $K_n$. Alcuni esempi di grafi completi sono

\disegna{

    \node[]  (A) at (2,0) {$K_2$};
    \node[nodo] (A1) at (-2,0){};
    \node[nodo] (A2) at (0,0){};
    \draw[] (A1) -- (A2);

    \node[]  (B) at (2,-3) {$K_3$};
    \node[nodo] (B1) at (-1,-2){};
    \node[nodo] (B2) at (-2,-4){};
    \node[nodo] (B3) at (0,-4){};
    \draw[] (B1) -- (B2);
    \draw[] (B2) -- (B3);
    \draw[] (B1) -- (B3);

    \node[]  (C) at (2,-7) {$K_4$};
    \node[nodo] (C1) at (-2.5,-6){};
    \node[nodo] (C2) at (-2.5,-8){};
    \node[nodo] (C3) at (0.5,-6){};
    \node[nodo] (C4) at (0.5,-8){};
    \draw[] (C1) -- (C2);
    \draw[] (C2) -- (C3);
    \draw[] (C1) -- (C3);
    \draw[] (C1) -- (C4);
    \draw[] (C3) -- (C4);
    \draw[] (C2) -- (C4);
}

\noindent 
Un grafo completo su $n$ vertici ha un numero di archi pari a 
$$\binom{n}{2} = \frac{(n)(n-1)}{2}$$
I grafi che consideriamo sono non orientati e \textbf{semplici}. Un grafo è semplice  se non ha loops (o cappi)

\disegna{
 \node[nodo](A){};
 \node at (1,0) (here){};
 \draw[->,>= stealth]  (A) edge [out=90,in=30,distance=10mm]   (A);
}

\noindent 
e non ha archi multipli, ovvero tra due nodi o c'è un arco non ce n'è neanche io. Quindi la situazione in figura non è ammessa.

\disegna{
 \node[nodo](A){};
 \node[nodo, right =of A] (B){};
 \draw[>= stealth]  (A) edge [out=60,in=150,distance=3.5mm]   (B);
 \draw[>= stealth]  (B) edge [out=-150,in=-60,distance=5mm]   (A);
}

\noindent 
Il sotto-grafo di un grafo $G = (V,E)$ è $G' =(V',E')$ tale che $V' \subseteq V$ e $E' \subseteq E \and [V']^2$. Nella seconda condizione imponiamo che se vogliamo avere l'arco $(i,j)$ nel grafo, allora $i,j \in V'$. Senza questa condizione non otterremmo un grafo. 

\begin{exmp}
    Dato il grafo 
    \disegna{
    \node[nodo] (C1) at (-2.5,-6){};
    \node[left=0.15cm] at (C1.east) {$1$}; 
    \node[nodo] (C2) at (-2.5,-8){};
    \node[left=0.15cm] at (C2.east) {$3$}; 
    \node[nodo] (C3) at (0.5,-6){};
    \node[right=0.15cm] at (C3.west) {$2$}; 
    \node[nodo] (C4) at (0.5,-8){};
    \node[right=0.15cm] at (C4.west) {$4$}; 
    \draw[] (C1) -- (C2);
    \draw[] (C1) -- (C3);
    \draw[] (C3) -- (C4);
    \draw[] (C2) -- (C4);
    }
    Se la seconda condizione fosse solamente $E' \subseteq E$ potremmo scegliere $V' = \{1,2\}$ ed $E' = \{(1,2), (2,3)\}$, ma siccome $3$ non è un nodo, il risultato non è un grafo.
    Un esempio di sotto-grafo è $V' = \{1,2,3,4\}$, $E' = \{(3,4)\}$

    \disegna{
    \node[nodo] (C1) at (-2.5,-6){};
    \node[left=0.15cm] at (C1.east) {$1$}; 
    \node[nodo] (C2) at (-2.5,-8){};
    \node[left=0.15cm] at (C2.east) {$3$}; 
    \node[nodo] (C3) at (0.5,-6){};
    \node[right=0.15cm] at (C3.west) {$2$}; 
    \node[nodo] (C4) at (0.5,-8){};
    \node[right=0.15cm] at (C4.west) {$4$}; 
    \draw[] (C2) -- (C4);
    }

    \noindent 
\end{exmp}

\noindent
Dato $V' \subseteq V$ il sotto-grafo $G'$ \textbf{indotto} da $V'$ è $G'(V',E')$ con $E' = E \and [V']^2$. Ovvero, se seleziono i vertici seleziono anche gli archi su cui sono incidenti. Dato l'insieme di vertici $V'$ c'è solo un sotto-grafo indotto.


Dato il grafo $G(V,E)$ il vicinato di $N(v)$ di $v \in v$ in $G$ è 

$$N(v) = \{j \in V: (v,j) \in E\}$$
Cioè tutti i nodi connessi a $v$ con un arco.

\begin{exmp}
Dato il grafo 
    \disegna{
    \node[nodo] (C1) at (-2.5,-6){};
    \node[left=0.15cm] at (C1.east) {$u$}; 
    \node[nodo] (C2) at (-2.5,-8){};
    \node[left=0.15cm] at (C2.east) {$v'$}; 
    \node[nodo] (C3) at (0.5,-6){};
    \node[above right=0.075cm] at (C3.north east) {$v$}; 
    \node[nodo] (C5) at (2.5,-6){};
    \node[nodo] (C4) at (0.5,-8){};
    \node[right=0.15cm] at (C4.west) {$\omega$}; 
    \draw[] (C1) -- (C2);
    \draw[] (C1) -- (C3);
    \draw[] (C3) -- (C4);
    \draw[] (C2) -- (C4);
    \draw[] (C5) -- (C3);
    \draw[] (C4) -- (C5);
    }
    Il vicinato di $v$ è $N(v) = V \backslash \{v\}$ mentre il vicinato di $v'$ è $N(v') = \{u,v,\omega\}$.
\end{exmp}

\noindent
Il grado di $v$ in $G$ è $d(v) = |N(v)|$. Se $v$ ha $d(v) = 0$ in $G$ allora si dice \textbf{isolato}.

\disegna{
   \node[]  (B) at (2,-4) {$ISOLATO$};
   \node[nodo] (I) at (2,-3){};
    \node[nodo] (B1) at (-1,-2){};
    \node[nodo] (B2) at (-2,-4){};
    \node[nodo] (B3) at (0,-4){};
    \draw[] (B1) -- (B2);
    \draw[] (B2) -- (B3);
    \draw[] (B1) -- (B3);
    \draw[->,>= stealth]  (B) edge [out=60,in=-60,distance=10mm]   (I);
}

\noindent 
Definiamo il grado minimo come $$\delta(G) = \min {d(v): v \in V}$$ e il grado massimo $$\Delta(G) = \max \{d(v) : v \in V\}$$ Se $\Delta(G) = \delta(G) = k$ allora $G$ è $k$-regolare.

\begin{exmp}
    Il seguente grafo è $2$-regolare
    \disegna{
    \node[nodo] (C1) at (-2.5,-6){};
    \node[nodo] (C2) at (-2.5,-8){};
    \node[nodo] (C3) at (0.5,-6){}; 
    \node[nodo] (C4) at (0.5,-8){};
    \node[nodo] (C5) at (-1,-5){};
    \node[nodo] (C6) at (-1.75,-9){};
    \draw[] (C1) -- (C2);
    \draw[] (C1) -- (C5);
    \draw[] (C5) -- (C3);
    \draw[] (C3) -- (C4);
    \draw[] (C2) -- (C6);
    \draw[] (C4) -- (C6);
    }
\end{exmp}

\noindent 
Il grado medio è
$$D(G) = \frac{1}{|V|} \sum_{v \in V} d(v)$$
Vale che $\delta(G) \leq D(G) \leq \Delta(G)$. La \textbf{densità} è invece definita come 

$$\varepsilon(G) = \frac{|E|}{|V|}$$
La densità ci dice quanti archi ha ,in media, ciascun vertice. Assomiglia al grado medio ma in quest'ultimo contiamo due volte ogni arco. Infatti vale che 

$$|E| = \frac{1}{2} \sum_{v \in V} d(v)$$
$$= \frac{1}{2} D(G) |V|$$
e quindi

$$\varepsilon(G) = \frac{|E|}{|V|}  = \frac{1}{2} D(G) $$

\begin{fatto}
In ogni grafo il numero di vertici di grado dispari è pari.
\end{fatto}

\begin{dimo}
    Cominciamo con l'osservare che $|E|$  è un numero intero, e siccome vale che  $|E| = \frac{1}{2} \sum_{v \in V}$ $d(v)$ allora anche $\frac{1}{2} \sum_{v \in V} d(v)$ è intero. Il valore $\sum_{v \in V} d(v)$ deve essere per forza pari, dato che la sua metà è intera. Dividiamo la sommatoria in due sommatorie:
\begin{align*}
\sum_{v \in V:\, d(v) \; \text{è pari}} d(v)  \\ + \sum_{v \in V:\, d(v) \; \text{è dispari}} d(v)
    \end{align*}
    La sommatoria pari ha come risultato sicuramente un numero pari. Questo vuol dire che, se come risultato finale vogliamo un numero pari, anche la sommatoria dispari deve risultare pari. Ciò è possibile se e solo se il numero di elementi è pari. Infatti, sommando un numero pari di numero dispari otteniamo un numero pari. Quindi il numero di vertici di grado dispari è pari.
\end{dimo}

\noindent 
Ci poniamo adesso la domanda se la densità può scendere sotto il grado minimo. Vediamolo prima con un esempio

\begin{exmp}
    Il seguente grafo 
    \disegna{
    \node[nodo] (C) at (-3,-2){};
    \node[nodo] (D) at (-1,-2){};
    \draw[] (C) -- (D);
    }

    \noindent 
    Ha $\delta(G) = 1$ e $\varepsilon(G) = \frac{1}{2}$, quindi $\delta(G) > \varepsilon(G)$
\end{exmp}

\begin{fatto}
    $\forall G$ con almeno un arco, ha un sotto-grafo indotto $H$ tale che 
    $$\delta(H) > \varepsilon(H) \geq \varepsilon(G) $$
\end{fatto}

\begin{dimo}
    Consideriamo una sequenza di grafi

    $$G = G_0, G_1, G_2, \dots$$
    Dove $G_i = (V_i,E_i)$ e $V_0 \supseteq V_1 \supseteq V_2$, con $G_i$ grafo indotto da $V_i$. Se $V_0(=V)$ ha $v_0$ tale che $d(v_0) \leq \varepsilon(G_0)$ creiamo $V_1 = V_0 \backslash \{v_0\}$. Notiamo che se non esiste $v_0$ che rispetta la condizione, allora 

    $$\forall v \in V d(v) > \varepsilon(G_0)$$
    e quindi $d(G_0) > \varepsilon(G_0)$. In questo caso avremmo già dimostrato il teorema con $H = G$.
    
    Consideriamo adesso $G_1$ indotto da $V_1$ (ricordiamo che $V_1 = V_0 \backslash v$). Iteriamo svolgendo la stessa operazione di prima fino a quando $V_i$ è tale che $\forall v \in V_i \, d(v) > \varepsilon(G_i)$. Notiamo che ci fermeremo prima di svuotare il grafo, infatti arriveremo al caso base 

        \disegna{
    \node[nodo] (C) at (-3,-2){};
    \node[nodo] (D) at (-1,-2){};
    \draw[] (C) -- (D);
    }

    \noindent
    dove sappiamo che vale $\delta(G) > \varepsilon(G)$. Se $G_{i+1}$ viene creato, allora

    $$\varepsilon(G_{i+1}) = \frac{|E_{i+1}|}{|V_{i+1}|}$$
    $$=  \frac{|E_i - d(v_i)|}{|V_i - 1|} \geq  \frac{|E_i - \varepsilon(G_i)|}{|V_i - 1|} $$
    Dove la disuguaglianza vale per la condizione con cui costruiamo il sotto-grafo. 
    $$= \frac{|E_i| - \frac{|E_i|}{|V_i|}}{|V_i - 1} =  \frac{|E_i| |V_i| - |E_i|}{|V_i|(|V_i - 1|)}$$
    dove abbiamo portato a fattore comune il numeratore.
    $$= \frac{|E_i| (|V_i| - 1)}{|V_i|(|V_i - 1|)} = \varepsilon(G_i)$$
    Quindi quando ci fermiamo avremo $G_k$ tale che $$\delta(G_k) > \varepsilon (G_k) \geq \varepsilon(G_0) $$

    
\end{dimo}

\chapter{Lezione II}

\noindent
Un \textbf{cammino} di lunghezza $k \geq 0$ in $G = (V,E)$ è un sotto-grafo $P_k$ con $k$ archi e $k+1$ vertici distinti tale che $e_i = (v_{i-1},v_i)$. Indichiamo gli archi con $e_1 \dots e_k$ e i nodi con $v_0,\dots,v_k$.

\disegna{
    \node[cloud,draw,minimum width = 5cm,
    minimum height = 4cm] {};
    \node[nodo] (A) at (1,1){}; 
    \node[] at(1,1.3) {$v_0$};
    \node[nodo] (B) at (1.5,0){}; 
    \node[nodo] (C) at (0.8,-1){}; 
    \node[nodo] (D) at (0,-0.6){}; 
    \node[nodo] (E) at (-2,-0.3){};
    \node[] at(-2,0) {$v_k$};
    \draw[] (A) -- (B) node[above,midway,sloped]{$e_1$};
    \draw[] (B) -- (C)  node[below right,midway]{$e_2$};;
    \draw[] (C) -- (D);
    \draw[] (D) -- (E) node[above,midway]{$e_k$};
}

\noindent 
Usiamo la nuvoletta quando non ci interessa la struttura del grafo. Evidenziamo solo una certa parte.
Nel caso in cui $P_0$ non abbiamo archi nel cammino ma un singolo vertice.

Un \textbf{ciclo} $C_k$ di lunghezza $k \geq 3$ è formato da un cammino $P_{k-1}$ che può essere esteso in $G$ includendo l'arco $(v_{k-1},v_0)$.


\disegna{
    \node[cloud,draw,minimum width = 5cm,
    minimum height = 4cm] {};
    \node[nodo] (A) at (1,1){}; 
    \node[] at(1,1.3) {$v_0$};
    \node[nodo] (B) at (1.5,0){}; 
    \node[nodo] (C) at (0.8,-1){}; 
    \node[nodo] (D) at (0,-0.6){}; 
    \node[nodo] (E) at (-2,-0.3){};
    \node[] at(-2,-0.6) {$v_{k-1}$};
    \draw[] (A) -- (B) node[above,midway,sloped]{$e_1$};
    \draw[] (B) -- (C)  node[below right,midway]{$e_2$};;
    \draw[] (C) -- (D);
    \draw[] (D) -- (E) node[above,near start]{$e_{k-1}$};
    \draw[color= red] (E) -- (A) node[above,midway,sloped,color = white] {$(v_{k-1},v_0)$};
}

\noindent 
In un grafo $G$, il \textbf{calibro} $g(G)$ è la lunghezza del ciclo più breve. La \textbf{circonferenza} è la lunghezza del ciclo più lungo. 

\begin{fatto}
    $\forall \; G$ con $\delta(G) > 2$ contiene un cammino di lunghezza $\delta(G)$ e un ciclo di lunghezza almeno $\delta(G) + 1$.
\end{fatto}

\begin{dimo}
    Prendiamo il cammino più lungo del grafo, $P_k$. Allora tutti i vicini di $P_k$ fanno parte del cammino, altrimenti potrei aggiungerli e allungarlo, $P_k$ non sarebbe il più lungo. Quindi il cammino $P_k$ è almeno lungo $|N(v_k)|$, dove $v_k$ è l'ultimo nodo del cammino. Siccome per ipotesi $|N(v_k)| \geq \delta(G)$ allora esiste un cammino di lunghezza  $\delta(G)$.
    Consideriamo ora il primo vertice che è un vicino di $v_k$.

    \disegna{
    \node[cloud,draw,minimum width = 5cm,
    minimum height = 4cm] {};
    \node[nodo] (A) at (1,1){}; 
    \node[] at(1,1.3) {$v_0$};
    \node[nodo, color = blue] (B) at (1.5,0){}; 
    \node[nodo] (C) at (0.8,-1){}; 
    \node[] at(1.8,0) {$v_i$};
    \node[nodo] (D) at (0,-0.6){}; 
    \node[nodo] (E) at (-2,-0.3){};
    \node[] at(-2,-0.6) {$v_{k}$};
    \draw[color = red] (A) -- (B);
    \draw[color = red] (B) -- (C);
    \draw[color = red] (C) -- (D);
    \draw[color = red] (D) -- (E);
    \draw[] (E) -- (B);
    \draw[] (E) edge [out=-60,in=-1200,distance=5mm] (C);
}
In rosso è evidenziato il cammino $P_k$ e in blu il primo vertice che è vicino di $v_k$. Se consideriamo il cammino in rosso da $v_i$ fino a $v_K$ e aggiungiamo $(v_k,v_i)$ troviamo un ciclo, ciò vale sempre per il fatto $\delta(G) \geq 2$. Il ciclo $C$ è lungo almeno $N(v_k) + 1 \geq \delta(G) + 1$.
    
\end{dimo}

\noindent
Dato $G = (V,E)$ $\forall i,j \in V \exists d(i,j)$ se $i,j$ sono connessi in $G$ da almeno $1$ cammino allora $d(i,j)$ è la lunghezza del cammino più breve, altrimenti è $\infty$. 

\begin{exmp}
Dato il grafo
\disegna{
     \node[cloud,draw,minimum width = 2.5cm,
    minimum height = 4cm] at(-2,0) {};
    \node[cloud,draw,minimum width = 2.8cm,
    minimum height = 4cm] at(2,0) {};

    \node[nodo] at(-1.8,0.3){};
    \node[nodo] at(2.3,-0.5){};
    \node[] at(-1.8,0.7) {$i$};
    \node[] at(2.3,-0.1) {$j$};
 }
 La distanza tra $i,j$ è $d(i,j) = \infty$.
\end{exmp}

\noindent
Il \textbf{diametro} è definito come

$$diam(G) = \max_{i,j \in V} d(i,j) =  \max_{i \in V}  \max_{j \in V} d(i,j) $$
e il raggio 

$$rad(G) = \min_{i\in V} \max_{j \in V} d(i,j)$$
Il raggio lo possiamo vedere come il punto "più centrale". Sia $x$ questo punto centrale, vale che $\forall v \in V d(x,v) \leq rad(G)$. Inoltre  $rad(G) \leq diam(G)$ e questo è ovvio dato che il diametro è una massimizzazione del massimo, mentre il raggio è una minimizzazione del massimo.
Possiamo anche dire che $diam(G) \leq 2 rad(g)$, dato che 
$$\forall u,v \in V d(u,v) \leq d(u,x) + d(x,v)$$
$$\leq rad(G) + rad(G) = 2 rad(G)$$

\begin{fatto}
    $\forall G$ che ha almeno un ciclo soddisfa

    $$g(G) \leq 2 diam(G) + 1$$
\end{fatto}

\noindent
\begin{dimo}
    Consideriamo il grafo
    \disegna{
    \node[cloud,draw,minimum width = 5cm,
    minimum height = 4cm] {};
    \node[nodo] (A) at (1,1){}; 
    \node[] at(1,1.3) {$x$};
    \node[nodo] (B) at (1.5,0){}; 
    \node[nodo] (C) at (0.8,-1){}; 
    \node[nodo] (D) at (0,-0.6){}; 
    \node[nodo] (E) at (-2,-0.3){};
    \node[nodo] (F) at (-1.4,0){};
    \node[nodo] (G) at (0,0.2){};
    \node[nodo] (H) at (0.2,1.2){};
    \node[] at(-2,-0.6) {$y$};
    \draw[color = red] (A) -- (B);
    \draw[color = red] (B) -- (C);
    \draw[color = red] (C) -- (D);
    \draw[color = red] (D) -- (E);
    \draw[color = blue] (E) -- (F);
    \draw[color = blue] (F) -- (G);
    \draw[color = blue] (G) -- (H);
    \draw[color = blue] (H) -- (A);
}
dove il ciclo $C$ è il più corto, con lunghezza $g(G)$. I due vertici $x,y$ sono vertici opposti, cioè tagliano il ciclo in due parti il più possibile uguali. Chiamiamo il percorso in rosso $p_1$ e il percorso in blu $p_2$. Assumiamo per assurdo che $g(G) \geq 2 diam(G) + 2$. Allora $p_1,p_2$ sono lunghi ciascuno almeno $diam(G) + 1$. Però $d(x,y) \leq diam(G)$ per la definizione stessa di diametro. Non tutti gli archi di $P$ (cioè del percorso più breve) stanno su $C$, altrimenti il ciclo avrebbe lunghezza $2diam(G) + 1$. Quindi, possiamo costruire un ciclo più piccolo, prendendo gli archi che non stanno né su $P_1$ né su $P_2$.

 \disegna{
    \node[cloud,draw,minimum width = 5cm,
    minimum height = 4cm] {};
    \node[nodo] (A) at (1,1){}; 
    \node[] at(1,1.3) {$x$};
    \node[nodo] (B) at (1.5,0){}; 
    \node[nodo] (C) at (0.8,-1){}; 
    \node[nodo] (D) at (0,-0.6){}; 
    \node[nodo] (E) at (-2,-0.3){};
    \node[nodo] (F) at (-1.4,0){};
    \node[nodo] (G) at (0,0.2){};
    \node[nodo] (H) at (0.2,1.2){};
    \node[] at(-2,-0.6) {$y$};
    \draw[color = red] (A) -- (B);
    \draw[color = orange] (B) -- (C);
    \draw[color = orange] (C) -- (D);
    \draw[color = orange] (D) -- (E);
    \draw[color = blue] (E) -- (F);
    \draw[color = blue] (F) -- (G);
    \draw[color = blue] (G) -- (H);
    \draw[color = blue] (H) -- (A);
    \draw[color = orange, dashed] (E) -- (B);
}

\noindent 
Il ciclo in arancione è più piccolo di $C$, quindi deve per forza valere che $g(G) \leq 2 diam(G) + 1$.
\end{dimo}

\section{Connettività di un grafo}
Un grafo è \textbf{sconnesso} se $\exists i,j \in V$ $|\; d(i,j) = \infty$. Una \textbf{componente} di un grafo è un qualunque insieme massimale di vertici connessi. Se un grafo è connesso il componente è il grafo stesso. $G$ è $k$-connesso se $|V| > k$ e $\forall X \subset V$ con $|X| < k$ il sotto-grafo indotto $V \backslash X$ è connesso.  Se un grafo è $k$-connesso non possiamo sconnettere il grafo rimuovendo al più $k-1$ vertici. Tutti i grafi sono $0$-connessi. Se $G$ è connesso è anche $1$-connesso, tranne il caso $K_1$ (cricca di un elemento) perché non rispetta la condizione $|V| > 1$. Il massimo intero $k$ tale che $G$ è k-connesso è detta \textbf{connettività} di $G$, che denotiamo con $K(G)$. Vale che $K(K_n) = n-1$.

\begin{exmp}
    Nel caso di $K_4$
    \disegna{
        \node[nodo] (C1) at (-2.5,-6){};
    \node[nodo] (C2) at (-2.5,-8){};
    \node[nodo] (C3) at (0.5,-6){};
    \node[nodo] (C4) at (0.5,-8){};
    \draw[] (C1) -- (C2);
    \draw[] (C2) -- (C3);
    \draw[] (C1) -- (C3);
    \draw[] (C1) -- (C4);
    \draw[] (C3) -- (C4);
    \draw[] (C2) -- (C4);
    }
    il numero di nodi che possiamo rimuovere è $3$.
\end{exmp}

\begin{teo}
    Se $G \notin \{k_0,k_1\}$ (ovvero $G$ non è un grafo banale), allora $K(G) \leq F \leq \delta(G)$ dove $k$ è qualsiasi insieme minimo di archi la cui rimozione sconnette il grafo.
\end{teo}

\begin{dimo}
    La disequazione $F \leq \delta(G)$ è banale. Infatti se sconnetto tutti gli archi attorno a un nodo ho sconnesso il grafo. Concentriamoci su $K(G) \leq F$ e distinguiamo due casi:

    \begin{itemize}
        \item $G$ ha un vertice $v$ che non è incidente a $F$.
    \end{itemize}
                    \disegna{
     \node[cloud,draw,minimum width = 2.5cm,
    minimum height = 4cm] at(-2,0) {};
    \node[cloud,draw,minimum width = 2.8cm,
    minimum height = 4cm] at(2,0) {};

    \node[nodo] (A) at(-1.6,0.9){};
    \node[nodo] (A1) at(-1.55,0.3){};
    \node[nodo] (A2) at(-1.65,-0.3){};
    \node[nodo] (A3) at(-1.6,-0.9){};
    \node[nodo] (v) at(-2.5,0){};
    \node[] at (-2.5,0.3){$v$};
    \node[nodo] (B) at(1.6,0.9){};
    \node[nodo] (B1) at(1.65,0.3){};
    \node[nodo] (B3) at(1.60,-0.9){};
    \draw[] (A)--(B1);
    \draw[] (A1)--(B);
    \draw[] (A2) -- (B3);
    \draw[] (A3) -- (B3);
    \draw (-1.5,0) ellipse (0.5cm and 2cm);
    \draw (0,0) ellipse (0.5cm and 1.5cm);
    \node[] (VC) at (-0.5,-2){$V_c$};
    \node[] (F) at (0,0){$F$};
    \node[] (C) at (-3,-2){$C$};
    \draw[->,>= stealth]  (VC) edge [out=-140,in=-60,distance=7mm]   (-1,-1.85);
     \draw[->,>= stealth]  (C) edge [out=80,in=--150,distance=7mm]   (-2.8,-1.5);
    }

    \noindent 

    dove $C$ è la componente del grafo che ottengo quando rimuovo $F$ e $V_c$ è l'insieme dei nodi connessi agli archi in $F$. Siccome rimuovendo $V_c$ sconnetto il grafo allora $K(G) \leq |V_c| \leq |F|$ 
    \begin{itemize}
        \item $G$ è tale che tutti i vertici sono incidenti con qualche arco in $F$.
    \end{itemize}
    \disegna{
     \node[nodo] (A1) at(-1.55,0.3){};
     \node[] at (-1.55,0.6){$v$};
    \node[nodo] (A2) at(-1.65,-0.3){};
    \node[nodo] (A3) at(-1.6,-0.9){};
    \node[nodo] (B) at(1.6,0.9){};
    \node[nodo] (B1) at(1.65,0.3){};
    \node[nodo] (B3) at(1.60,-0.9){};
    \draw[] (A1) -- (B1);
    \draw[] (A2) -- (B);
    \draw[] (A3) -- (B3);
    \draw[>= stealth]  (A1) edge [out=120,in=-120,distance=7mm]   (A3);
    \draw[>= stealth]  (B3) edge [out=60,in=-60,distance=7mm]   (B);
    \draw[] (A1) -- (A2);
    \draw[] (A3) -- (A2);
    \draw[] (B) -- (B1);
    \draw[] (B1) -- (B3);
    \draw (0,0) ellipse (0.5cm and 1.5cm);
    \node[] (F) at (0,0){$F$};
    }
    Il grafo $G$ ha connettività $K(G) \leq d(v)$. Siccome $d(v) = |F| = \delta(G)$ vale che $K(G) \leq |F|$.
\end{dimo}

\chapter{Terza lezione}


\noindent
Un cammino \textbf{chiuso} (in inglese closed walk) è un ciclo in cui i vertici non sono distinti. Un cammino chiuso si dice \textbf{euleriano} se attraversa tutti gli archi del grafo esattaemente una volta. Un grafo è euleriano se ammette un cammino euleriano.

\begin{teo} \textbf{Teorema di Eulero (1746)}

\noindent 
Un grafo connesso è euleriano se e solo se ogni vertice ha grado pari.
\end{teo}

\begin{dimo}
    Cominciamo con dimostrare il lato $=>$ del teorema. Quindi, dato un grafo connesso euleriano questo ogni vertice ha grado pari. Prendiamo un vertice che si trova sul cammino euleriano.
    \disegna{
        \node[nodo] (X) {};
        \node[nodo] (A) at (1,1.5) {};
        \node[nodo] (B) at (1,-1.5) {};
        \node[nodo] (C) at (-1,1.5) {};
        \node[nodo] (D) at (-1,-1.5) {};
        \node[nodo] (E) at (-2,0.75) {};
        \draw[>= stealth]  (X) edge [out=60,in=-60,distance=7mm]   (A);
        \draw[>= stealth]  (X) edge [out=-30,in=110,distance=7mm]   (B);
        \draw[>= stealth]  (X) edge [out=110,in=-30,distance=7mm]   (C);
        \draw[>= stealth]  (X) edge [out=-100,in=60,distance=7mm]   (D);
        \draw[>= stealth, dotted]  (X) edge [out=120,in=60,distance=7mm]   (E);
    }
    \noindent
    Se il cammino passa per il vertice, allora deve sia entrare che uscire. Non può esserci un arco che collega un vicino che non sia nel cammino. Quindi o un vertice è isolato oppure il cammino esce ed entra. Allora devono avere grado pari.
    
    L'altro verso necessita un po' più di lavoro per essere dimostrato. Quello che vogliamo dimostrare è che se ogni vertice ha grado pari allora il grafo è euleriano.  Facciamo una dimostrazione per induzione su $|E|$.

    \noindent 
    \textbf{Caso base} $|E| = 0$, banale. Implica che $|V| = 1$ perché parliamo di grafi connessi.

    \noindent 
    \textbf{Ipotesi induttiva} $|E| \geq  1$. Enunciamo un fatto utile.

    \begin{fatto}
        Se $G$ ha tutti i vertici con grado pari con $E \leq 1$, posso trovare in $G$ un cammino chiuso che non contiene un arco \textbf{più} di una volta. 
    \end{fatto}

    \noindent
    Sia $\omega$ un tale cammino di lunghezza massima. Ne rappresentiamo uno da esempio in figura.
    
    \disegna{
        \node[nodo] (A) at (0,0) {};
        \node[nodo] (B) at (1,1) {};
        \node[nodo] (C) at (2,0) {};
        \node[nodo] (D) at (1,-1) {};
        \node[nodo] (E) at (-1,1) {};
        \node[nodo] (F) at (-1,-1) {};
        \node[nodo] (G) at (-2,0) {};

        \draw[] (A) -- (B);
        \draw[] (A) -- (E);
        \draw[] (A) -- (D);
        \draw[] (A) -- (F);
        \draw[] (B) -- (C);
        \draw[] (C) -- (D);
        \draw[] (G) -- (E);
        \draw[] (G) -- (F);
    }

    \noindent 
    Definiamo come $F$ l'insieme degli archi di $\omega$. Se $F \equiv E$ allora abbiamo finito, dato che tutti gli archi di $G$ fanno parte del cammino $\omega$. Assumiamo per assurdo che non sia così. Allora deve valere $$E' \equiv E \backslash F \neq \varnothing$$ Quindi 
\end{dimo}

\noindent 
Se un grafo è euleriano possiamo trovare un cammino euleriano in tempo $O(|E|)$, i.e. in tempo lineare nella descrizione del grafo (algoritmo di Hierholzer)..

\end{document}
